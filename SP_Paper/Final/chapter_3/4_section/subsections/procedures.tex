\subsection{Procedures}
\label{subsec:procedures}
This section provides a general overview of the steps conducted
in the development of the different components developed in the
whole duration of this special problem.

% Procedure for the Development of the alamSYS and its Components
\subsubsection{Procedures for the Development of the alamSYS and its Components}
\label{subsubsec:proc_alamSYS}
The following step-by-step procedures were conducted in the development
of the alamSYS and its components (alamAPI, alamDB, and alamPREPROCESSOR):
\begin{itemize}
    \item[(a)] User Requirement and Analysis
    \item[(b)] System Design and Architecture using DFD
    \item[(c)] System Prototyping and Development.
    \item[(d)] System Testing and Continuous Development and Integration.
    Each of the components of the alamSYS were developed into smaller chunks,
    and each chunks were tested and integrated with the other smaller chunks
    until a component was functionally developed. Afterwards the same process for
    the other components were conducted, until a whole working system was developed.
    \item[(e)] System Deployment, Review, and Maintenance
    \item[(f)] Steps (a) to (e) were repeated until the system was deemed
    to be stable and ready for deployment within the scopes of this special
    problem. Moreover, the system shall be maintained even after the submission
    of this special problem such as the addition of new features.
\end{itemize}
\hfill \\


% Procedure for the Development of the DMD-LSTM Model
\subsubsection{Procedures for the Development of the DMD-LSTM Model}
\label{subsubsec:proc_dmdlstm}
The following step-by-step procedures were conducted in the development
of the DMD-LSTM model:
\begin{itemize}
    \item[(a)] Collection of end-of-day Philippine Stock Market Data from the
    20 selected stocks from the Philippine Stock Exchange (PSE) for the periods
    until February 10, 2023.
    \item[(b)] Data Preprocessing, Model Training, Testing, Cross-validation, and Evaluation
    was conducted based on the DMD-LSTM methodology shown in Figure \ref{fig:ml_model}.
    \item[(c)] The best performing DMD-LSTM model was deployed and integrated as part 
    of the alamSYS, specifically used in the alamPREPROCESSOR component.
\end{itemize}
\hfill \\


% Procedure for the Development of the ALMACD Trading Algorithm
\subsubsection{Procedures for the Development of the ALMACD Trading Algorithm}
\label{subsubsec:proc_almacd}
The following step-by-step procedures were conducted in the development
of the ALMACD trading algorithm:
\begin{itemize}
    \item[(a)] Collection of end-of-day Philippine Stock Market Data from the
    20 selected stocks from the Philippine Stock Exchange (PSE) for the periods
    until February 10, 2023.
    \item[(b)] Using a Grid Search approach, the best ALMA parameters 
    (\textit{i.e.} window size, sigma, and offset) for both slow and fast were
    determined.
    \item[(c)] Using the best ALMA parameters, a trading algorithm was developed
    which takes in the convergence and divergence of the slow and fast ALMA as basis
    for entry and exit signals. When the slow ALMA crosses above the fast ALMA, a
    buy signal is generated. When the slow ALMA crosses below the fast ALMA, a sell
    signal is generated.
    \item[(d)] The trading algorithm was deployed and integrated as part of the
    alamSYS, specifically used in the alamPREPROCESSOR component.
\end{itemize}
\hfill \\


% Procedure for the Development of Mobile-based Test Application
\subsubsection{Procedures for the Development of Mobile-based Test Application}
\label{subsubsec:proc_mobdev}
The following step-by-step procedures were conducted in the development
of the mobile-based test application:
\begin{itemize}
    \item[(a)] User Interface Design and Development using Figma.
    \item[(b)] Creation of Initial Prototype using Flutter.
    \item[(c)] Continuation of Feature Development and Testing.
    \item[(d)] Integration of HTTP methods with the alamAPI.
\end{itemize}
\hfill \\
\textit{Note that the purpose of this application is to showcase the main feature of
the alamSYS, as such it is not fit for user deployment. However it is further discussed
in the recommendations on how it can be potentially deployed by future developers.}

% Procedure for the System Testing
\subsubsection{Procedures for the System Testing}
\label{subsubsec:proc_mobdev}
The following step-by-step procedures were conducted in the system testing:
\begin{itemize}
    \item[(a)] Development of tester application, which are as follows:
    \begin{itemize}
        \item System Statistics Logger - Using the docker stats stream command, the system 
        CPU, memory, and network utilization stats were recorded. This logger was used
        to determine the idle and on load performance of the alamSYS.
        \item Deployment Tester - This tester application was used to test the deployment
        reliability of the alamSYS. The test was conducted by deploying the alamSYS in a server
        while then other computers requests for different functions to the alamSYS over the internet
        using the HTTP methods. Further the tester also logs the overall response time to
        process 10, 100, and 1000 requests in a row, as well as the success rate of the
        requests.
        \item Internal System Stress Tester - This tester application was used to test the
        internal system reliability of the alamSYS. The test was conducted to run the processes
        of the alamSYS in a loop for 100 consecutive times, and logs the overall response time
        as well as success rate for each processes.
    \end{itemize}
    \item[(b)] Initial testing of the tester applications, to see if
    they are working as intended
    \item[(c)] Actual testing was done using these tester applications, and
    all results and system logs were recorded.
    \item[(d)] The results of the testing were then analyzed, compared, and summarized.
    Further details regarding this are discussed on Chapter 4.
\end{itemize}
\hfill \\