\section{ALMACD Results and Discussions}
\label{sec:almacd_res}
The ability to predict consecutive days in the stock market is useless 
without a trading strategy - which allows risk mitigation and increases 
the probability of positive returns over time. Trading strategies, in 
particular, are based on a predefined set of rules and criteria that 
are used to determine when to buy and sell stocks.
\cite{Hayes2022}.
\\

A variety of algorithmic trading, on the other hand, refers to the use of 
mathematical and computational techniques to determine the best position to 
take for a specific set of stocks. Additionally, the possibility of loss 
due to the influence of human emotion is eliminated.
\cite{WallStreetMojo}.
\\

Whereas, the author used the Arnaud Legoux Moving Average Convergence and Divergence 
(ALMACD) trading strategy in this special problem and integrated it into the 
alamSYS as the system's internal trading algorithm. ALAMCD uses predicted 
prices for the next 5 days, as well as 200 days of actual stock price data, 
to track the signals and output a simple flag indicating whether to buy or 
sell that stock at that time.
\\

The compounded expected return after return backtesting is 
provided for each stock in Table \ref{tab:optimal_alma_validation}
using the optimized parameters for the fast and slow ALMA. 
This was done to validate the potential returns for all stocks, not just the 
PSEI, from which the best ALMA parameters were derived.
\\

\begin{longtable}[c]{cc}
    \caption{Optimal Alma Parameters Validation Results}
    \label{tab:optimal_alma_validation}\\
    \hline
    \textbf{Stock} & \textbf{Compounded Expected Return} \\ \hline
    \endfirsthead
    %
    \multicolumn{2}{c}%
    {{\bfseries Table \thetable\ continued from previous page}} \\
    \hline
    \textbf{Stock} & \textbf{Compounded Expected Return} \\ \hline
    \endhead
    %
    \hline
    \endfoot
    %
    \endlastfoot
    %
    \textbf{PSEI}  & 113966.8500                         \\
    \textbf{AC}    & 20893.1914                          \\
    \textbf{ALI}   & 1072.1418                           \\
    \textbf{AP}    & 690.7100                            \\
    \textbf{BDO}   & 2541.9970                           \\
    \textbf{BLOOM} & 495.4600                            \\
    \textbf{FGEN}  & 581.0804                            \\
    \textbf{GLO}   & 60538.0035                          \\
    \textbf{ICT}   & 2815.6103                           \\
    \textbf{JGS}   & 1569.8650                           \\
    \textbf{LTG}   & 397.2854                            \\
    \textbf{MEG}   & 149.2233                            \\
    \textbf{MER}   & 8586.0306                           \\
    \textbf{MPI}   & 146.0200                            \\
    \textbf{PGOLD} & 721.2700                            \\
    \textbf{RLC}   & 649.4767                            \\
    \textbf{RRHI}  & 1050.7000                           \\
    \textbf{SMC}   & 2557.0770                           \\
    \textbf{TEL}   & 72070.5000                          \\
    \textbf{URC}   & 3207.5394                           \\ \hline
\end{longtable}}
Based on the table of expected returns above, all stocks are expected to 
return a positive yield over time when these optimal ALMA parameters are used. 
It is also worth noting that the expected return is calculated for each unit 
of stock, which means that if we use the expected compounded return value of 
MPI at PHP 146.02, which appears to be the lowest - the actual return could 
be at least PHP 146,020, assuming the minimum board lot required for the stocks 
is 1000 shares \cite{Pesobility}.
\\

However, despite the high potential returns, investors should proceed with 
caution for two reasons. First, the expected return is based on historical 
price data, which may not follow the trend of future price data, potentially 
rendering the trading algorithm obsolete \cite{QuantifiedStrategies}. 
Second, the return calculation does not account for and compensate for the 
additional fees associated with buying and selling the stock, which can 
affect the overall actual returns.
\\