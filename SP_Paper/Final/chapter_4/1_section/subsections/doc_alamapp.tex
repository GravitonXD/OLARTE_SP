\subsection{Documentation for alamAPP}
\label{subsec:doc_alamAPP}
The alamAPP is a mobile-based test application developed to 
demonstrate the use of the alamSYS to external internet-connected 
devices, such as a smartphone application. It was created with the 
Dart-based Flutter SDK. The user interface and code snippets of how 
the alamSYS was used in the alamAPP are shown in this section. 
Furthermore, the full source code for the alamAPP is available 
in the GitHub repository listed in Appendix A of this paper.

\subsubsection{User Interface Showcase}
\label{subsubsec:alamAPP_UI}
This section displays the alamAPP's developed user interfaces. 
It should be noted that this application only highlights the 
alamSYS's two most important features, which are its ability 
to recommend which stocks to buy and which to sell. Additional 
enhancements to the application are discussed in Chapter 5 of 
this paper.
\\

% Insert Buy UI Here 
xxx
\\

% Insert Sell UI Here
xxx
\\

% Insert UI when a card is selected
xxx
\\

% Insert UI when the application is retrieving data from the alamAPI
xxx
\\

% Insert UI when it encountered an error
xxx
\\

Overall, the figures depicting the alamAPP's user interface 
demonstrate that the alamAPI can be used by external applications 
by connecting via the internet.

\subsubsection{Utilization of alamSYS in the alamAPP}
\label{subsubsec:utilization_alamSYS-alamAPP}
The following code snippet written in Dart, was used in the alamAPP's source 
code to get the details of the stocks to buy from alamAPI.
\textit{Note that the [reserved.domain] should be changed to
the domain address used on the alamAPI deployment}.
\hfill \\
\begin{python}
    Future<List<Stocks>> stocksToBuy() async {
        var url = 'https://[reserved.domain]/stocks_to_buy/all';
        var response = await http.get(Uri.parse(url));
        var data = jsonDecode(response.body);
        List<Stocks> stocksToBuy = [];
        if (data['Stocks'] != null) {
        data['Stocks'].forEach((v) {
            stocksToBuy.add(new Stocks.fromJson(v));
          });
        }
        return stocksToBuy;
    }
\end{python}

Meanwhile for the stocks to sell, the following code
snippet was used:
\hfill \\
\begin{python}
Future<List<Stocks>> stocksToSell() async {
      var url = 'https://[reserved.domain]/stocks_to_sell/all';
      var response = await http.get(Uri.parse(url));
      var data = jsonDecode(response.body);
      List<Stocks> stocksToSell = [];
      if (data['Stocks'] != null) {
        data['Stocks'].forEach((v) {
          stocksToSell.add(new Stocks.fromJson(v));
        });
      }
      return stocksToSell;
    }
\end{python}
