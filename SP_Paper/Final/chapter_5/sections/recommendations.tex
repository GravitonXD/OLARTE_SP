\section{Recommendations on Future Works}
\label{sec:recommendations}
This section presents a road map for future research based 
on the insights gained in the preceding sections of this chapter. 
The roadmap outlines key areas for additional research, potential 
system enhancements, opportunities for integration with external 
applications or systems, and other applicable considerations.

\begin{itemize}
    \item[(a)] In connection to the development of the alamSYS:
        \begin{itemize}
            \item[1.] Automated Model and Trading Algorithm Evaluator and Trainer - 
            Given the possibility of changes in market price trend data 
            over time, the patterns on which the current DMD-LSTM is based may 
            become obsolete. As a result, it may be advantageous to incorporate 
            an additional component that would revalidate the deep learning 
            model as well as the trading algorithm, and if it performs worse 
            than 10\% of the current performance, it would automatically 
            rerun the training with the new data until it finds a model and 
            optimal parameters that performs inline or even better than the 
            current one.
            \item[2.] Alternative System Deployment Methods can be Explored - 
            The tunneling service was found to be responsible for the 
            majority of the delay in the alamSYS response time after deployment. 
            As a result, it may be beneficial to investigate better alternatives, 
            such as deploying the system on a server with a fast and reliable 
            internet connection, though this may necessitate a larger budget. 
            Another low-cost option is to deploy the system on cloud computing 
            platforms such as AWS and Google Cloud.
            \item[3.] Integration of other Models and Trading Algorithm - 
            Initially, it was planned to integrate multiple combinations of 
            machine learning models and trading algorithms into alamSYS, but due 
            to time constraints, this was not pursued. However, the system was 
            designed with this in mind, and the source code already allows for 
            this feature to be integrated. Future developers, for example, can 
            investigate the integration of ARIMA-based models in alamSYS. This 
            provides users with multiple sources of data-driven information on 
            the market's stocks. Where their decision is not limited to a single 
            piece of information.
            \item[4.] A Closer Examine on the Stock Suggestions of the System - 
            Because the alamSYS currently only logs its daily predictions, it may 
            be useful for future research to examine the trend of these predictions 
            and try to see the accuracy of this suggestion, as well as potentially 
            add improvements to the system; and
            \item[5.] System Maintenance and Additional Features or Improvements
            to the Current System - 
            The completion of this paper does not indicate the end of the alamSYS's 
            development lifetime; it is expected that the developer, and other 
            developers in the future, will continue to work on improving the system 
            from its initial deployment, using the continuous integration, continuous 
            delivery, and deployment - CI/CD pipeline \cite{RedHat2022}.
        \end{itemize}

    \item[(b)] In connection to the development of DMD-LSTM Model, and ALMACD Trading Algorithm:
        \begin{itemize}
            \item[1.] Solving for the Observed Prediction Offset - 
            The figures comparing predicted and actual price data for each stock 
            revealed a distinct offset pattern. With this in mind, it may be 
            worthwhile to investigate further and add potential additional steps 
            to the predictions to compensate for the offset and improve the model's 
            performance. This could be done during training or as a post-processing 
            technique.
            \item[2.] Exploration of Other Machine Leaning Models - 
            There are numerous machine learning models that can be investigated for 
            their potential to predict stock market movement. As a result, it is 
            recommended that future developers or researchers be able to devise novel 
            approaches to creating and integrating other models based on the findings 
            and limitations of this special problem.
            \item[3.] Redefine the Optimal Fast, and Slow ALMA Parameters - 
            The current optimal ALMA parameters, which do not account for the 
            additional fees for buying and selling stock positions, could possibly 
            not yield to the highest return potentials. As a result, future developers 
            must investigate the possibility of including these fees from the 
            Philippine Stock Exchange in order to better compute the optimal 
            ALMA parameters, and then compare this results on the results of the
            current optimized ALMA parameters deployed in the alamSYS; and
            \item[4.] Exploration of Other Trading Strategies and Algorithms - 
            The stock market strategies is not confined on the utilization of moving 
            averages such as Arnaud Legoux Moving Average in creating simple yet 
            effective trading strategies. Hence, it would be beneficial if other trading 
            strategies were explored being integrated alongside the machine learning 
            models to be developed as well.
        \end{itemize}

    \item[(c)] In connection to the development of the mobile-based test application, 
    future developers are encouraged to use other internet-connected or 
    Internet-of-Things (IoT) devices such as smart speakers, smartwatches, 
    smart glasses, and so on. To create an application that makes effective 
    use of the alamSYS. Making certain that it is widely used.
\end{itemize}