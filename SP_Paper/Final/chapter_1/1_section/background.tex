% Background and Rationale
\section{Background and Rationale}
\label{sec:background}
In recent years, 'artificial intelligence' (AI) has grown in popularity as a tool 
for nearly every endeavor. Particularly in recent years, machine learning techniques 
are increasingly being utilized in the stock market, specifically for: 
(a) stock price predictions; (b) stock movement/trend prediction; and 
(c) stock portfolio management
\cite{Kumbure2022, Strader2020, Soni2022, Rea2020, Guo2022, Budiharto2021}. 

However, the utilization of these tools has its limitations. First, these tools
are only based on the past market data, and may not be able to take account
for other events and technical aspects, that can affect the stock market
\cite{Concoda2020}. 
Second, most of these tools are closed off to the public, or often behind a 
payment or subscription walls, limiting the resources available for them.
Lastly, there are no existing integrations of these developed machine learning
models with other traditional stock market analysis tools, making them somewhat
ineffective when utilized in a real-world scenario
\cite{CHHAJER2022100015, zou2023stock}.

These points are essential to take note, since investing in the stock market brings 
significant financial benefits to individuals such as: 
(a) it helps alleviate the effects of inflation over time. Whereas, it was noted 
that the Philippines' inflation rate as of April 2023 was 6.6\% \cite{NEDA2023}, 
while savings account  deposit interest rates are only 1-3\% annually \cite{BSP19}. 
Meaning, savings in deposit banks may not keep pace with inflation, 
potentially reducing an individual's purchasing power over time \cite{RBC, EdwardJones}; 
and (b) investing in the stock market can provide individuals with the opportunity for 
significant capital growth without the need for direct investment involvement in 
business operations \cite{USSecAndExComm}.

Also, the stock market is widely acknowledged to play an important role in economic 
growth because it allocates and provides capital to businesses, which drives economic 
activity and growth. This is evident from the fact that stock market performance 
is frequently correlated with the gross domestic product (GDP) of the country
\cite{TradeBrains, Hall2022, Bae2017}.
Furthermore, historical stock price trends can provide insight 
into broader economic movements
\cite{Campbell2021}. 
In a study conducted by \citeA{Balaba2017}, they discovered that the stock market
has a positive impact on the Philippines' economy. The study's findings showed that
as the stock market rose, the unemployment rate fell. This is because the performance 
of the stock market leads to job creation, which in turn leads to economic growth.
This, in turn, drives economic growth. This relationship was observed
in the Philippines from 2007 to 2017.


Whereby, having access to these cutting-edge tools, can improve how the general public interacts
with the stock market. Hence this special problem was completed, in order to 
explore and improve the utilization of machine learning models in the 
Philippine Stock Market, particularly deep learning models such as LSTMs through
the development of the alamSYS.