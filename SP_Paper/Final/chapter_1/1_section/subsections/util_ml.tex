% Utilization of Machine Learning in Stock Market Trading
\subsection{Utilization of Machine Learning in Stock Market Trading}
In recent years, there has been a growing interest in applying machine 
learning techniques to predict the movement of the stock market, 
both in the short and long term. This has led to numerous studies and 
practical applications exploring the use of machine learning in stock market prediction.
 These efforts aim to improve the accuracy of predictions and help investors make 
 informed decisions.
\cite{Kumbure2022, Strader2020, Soni2022, Rea2020, Guo2022}.
 Wherein, one of the common techniques used in machine learning for 
 stock market prediction is Long Short-Term Memory (LSTM). 
 A study by \citeA{Budiharto2021} found that LSTM was effective in predicting the 
 Indonesian stock market by 95\% using short-term data. 
 This indicates that LSTM can be a valuable tool for making short-term stock 
 market predictions.
 \hfill \\

Recently, the use of Dynamic Mode Decomposition (DMD) 
for predicting stock market price trends has gained momentum in the financial 
and scientific communities. DMD is a mathematical method that can be used 
to identify patterns and trends in complex data sets, such as stock market data.
By applying DMD to stock market data, it is possible to make more accurate 
predictions about future stock price movements. This can help investors make
informed decisions about their investments and potentially generate better returns.
In connection to this, a study by \citeA{Lu2020} found that DMD can be a 
faster predictor than Proper Orthogonal Decomposition (POD), but 
it is less accurate.
\hfill \\

Furthermore, other studies have shown that DMD can be effectively 
applied to the Turkish and Indian stock markets to predict market price trends
\cite{Savas2017, Kuttichira2017}.
These studies indicate that DMD is easy to implement and can be a useful 
tool for making stock market predictions.
\hfill \\

Aside from LSTM and DMD, another model is also being used in stock market predictions, 
which is the Auto Regression Integrated Moving Average (ARIMA). 
In a study conducted by \citeA{Adebiyi2014}, 
ARIMA model shows satisfactory results for predicting stock prices on the short-term period. 
Moreover, in this special problem the author will also explore the feasibility of using 
Arnaud Legoux Moving Average (ALMA) in combination with ARIMA. This will be done since 
compared to the traditional Moving Average (MA), 
ALMA produces a more reliable signal 
\cite{Sarkar2019}.