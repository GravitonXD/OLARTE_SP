\chapter{Introduction}
\label{sec:researchdesc}

% Background and Rationale
\section{Background and Rationale}
\label{sec:background}
The stock market is a type of market that allows companies to raise capital 
by issuing shares of stock to investors. These shares represent a share of 
ownership in the company and entitle the holder to a share of the company's 
profits and voting rights. The stock market also provides a platform for 
investors to buy and sell these shares, allowing for the efficient trading of 
company ownership. By allowing companies to raise capital and investors to buy and 
sell shares, the stock market plays a crucial role in the growth and development 
of the economy
\cite{Chen2022, TheEconomicTimes}.
\\Contrary to popular belief, the stock market is not a form of gambling. 
It involves a significant amount of analytical thinking and risk management, 
and the returns are based on the supply and demand for a given stock, rather 
than on false promises or assurances. In other words, the stock market is a 
legitimate platform for investing and generating returns, rather than a scam or 
gamble
\cite{Schwab-Pomerantz2021,Adams2022,Summers2022}.

% The Philippine Stock Exchange (PSE)
\subsection{The Philippine Stock Exchange (PSE)}
The Philippine Stock Exchange (PSE), Inc. is the official stock exchange 
market in the Philippines. It is a non-stock company that was incorporated 
in 1992 and manages and operates the stock market in the country. Registered 
individuals can participate in market exchanges on the PSE.
\cite{PSECompanyInfo}.
\\Moreover, the main index of the Philippine Stock Exchange (PSE) 
is the Philippine Stock Exchange Index (PSEI). The PSEI is a market 
capitalization-weighted price index that is based on the 30 largest and 
most actively traded companies on the PSE. These companies are pre-determined 
based on strict criteria, such as liquidity and market capitalization. 
The PSEI is often used as a benchmark for the performance of the overall stock 
market in the Philippines.  
\cite{BSP}
The companies that make up the PSEI are often referred to as blue-chip companies, 
as they are typically large, well-established companies with a history of strong 
financial performance. As of October 2022, there are 286 companies listed on the PSE,
 providing a diverse range of investment opportunities for investors. 
\cite{Fayed2022, PSECompanyList}.

% Economic Relevance and Benefits of Stock Market Investment
\subsection{Economic Relevance and Benefits of Stock Market Investment}
It is commonly accepted that the stock market plays a crucial role in economic growth,
 as it allocates and provides capital to businesses, which in turn drives economic
  activity and growth. This is evident from the fact that stock market performance
   is often correlated with a country's Gross Domestic Product (GDP)  
  \cite{TradeBrains, Hall2022, Bae2017}
  Additionally, historical trends in stock prices can provide insight into 
  broader economic movements 
\cite{Campbell2021}.
\\Moreover, a study by \citeA{Balaba2017} found that the stock market has a 
positive effect on the economy of the Philippines. The data from the study 
showed that as the stock market grew, the country's unemployment rate declined. 
This is because the stock market's performance leads to job creation, which in 
turn drives economic growth. This relationship has been evident in the Philippines 
for the past 10 years.

% Benefits of Invesitng for the Individual
\subsection{Benefits of Investing for the Individual}
The Philippine Stock Exchange allows individuals in the Philippines
 to trade shares of listed companies. Investing in the stock market can provide 
 several benefits for an individual, such as:
\begin{itemize}
    \item[(a)] Protecting the value of an individual's money from inflation: Inflation 
    in the Philippines was at 6.9\% as of September 2022 
    \cite{tradingEconomics}, 
    while savings account deposit interest rates are only between 1-3\% annually 
    \cite{BSP19}. 
    This means that savings in deposit banks may not keep pace with inflation, 
    potentially reducing the purchasing power of an individual's money 
    \cite{RBC, EdwardJones}.
    \item[(b)] Providing opportunities for capital growth: 
    Investing in the stock market can provide individuals with the p
    otential for significant capital growth, without the need for direct 
    involvement in business operations. This can be beneficial for individuals
    such as students or working professionals, who can grow their capital while 
    focusing on their studies or careers
    \cite{USSecAndExComm}.
\end{itemize}

% Utilization of Machine Learning in Stock Market Trading
\subsection{Utilization of Machine Learning in Stock Market Trading}
In recent years, there has been a growing interest in applying machine 
learning techniques to predict the movement of the stock market, 
both in the short and long term. This has led to numerous studies and 
practical applications exploring the use of machine learning in stock market prediction.
 These efforts aim to improve the accuracy of predictions and help investors make 
 informed decisions.
\cite{Kumbure2022, Strader2020, Soni2022, Rea2020, Guo2022}.
 Wherein, one of the common techniques used in machine learning for 
 stock market prediction is Long Short-Term Memory (LSTM). 
 A study by \citeA{Budiharto2021} found that LSTM was effective in predicting the 
 Indonesian stock market by 95\% using short-term data. 
 This indicates that LSTM can be a valuable tool for making short-term stock 
 market predictions.
\\Recently, the use of Dynamic Mode Decomposition (DMD) 
for predicting stock market price trends has gained momentum in the financial 
and scientific communities. DMD is a mathematical method that can be used 
to identify patterns and trends in complex data sets, such as stock market data.
By applying DMD to stock market data, it is possible to make more accurate 
predictions about future stock price movements. This can help investors make
informed decisions about their investments and potentially generate better returns.
In connection to this, a study by \citeA{Lu2020} found that DMD can be a 
faster predictor than Proper Orthogonal Decomposition (POD), but 
it is less accurate.
\\Furthermore, other studies have shown that DMD can be effectively 
applied to the Turkish and Indian stock markets to predict market price trends
\cite{Savas2017, Kuttichira2017}.
These studies indicate that DMD is easy to implement and can be a useful 
tool for making stock market predictions.

% Statement of the Problem
\section{Statement of the Problem}
\label{sec:problem}
The Philippines' economic growth is expected to decline in the coming 
years due to the global pandemic, high inflation, and low employment rates. 
\cite{Alegado2022,Canto2022,Reuters2022}.
\\Currently, the lack of free and publicly available stock market predictive systems or
 tools creates a gap in the information available to the public compared to large private 
 individuals or institutions. These large institutions have the resources to spend a significant 
 amount of money on stock market research, giving them a significant advantage in the investing 
 market. This lack of access to the same information puts the public at a disadvantage
 \cite{Kim2022}. 
\\ Furthermore, the lack of publicly available stock market prediction tools can 
lead to unwise investment decisions by individuals, particularly first-time investors, 
resulting in significant losses and discouragement from investing in the stock market. 
This is a significant problem, as the number of local investors in the Philippine Stock Market 
is already quite low, comprising only around 1\% of 
the total population and there also has been a massive decline in foreign investment
 in the Philippines in recent years 
 \cite{BusinessWorld2022}, 
 leading to a corresponding decline in investment volume. 
 As suggested by the study of \citeA{Balaba2017}, 
 this is expected to have a negative multiplier effect on the country's 
 economic development in the future. 
\\ Hence, the development of a publicly available, easy-to-use, and
 accurate stock market price trend prediction system could help to 
 reduce the information gap and level the playing field for individual investors. 
 By providing the public with fast and reliable information, this system could help to
  increase transparency and fairness in the stock market, leading to more informed 
  and confident investing decisions and ultimately a more stable and prosperous market.
   Additionally, such a system could help to increase the participation of individual 
   investors in the market, leading to a more diverse and stable market overall.
\cite{Statista2022,POPCOMM2021}.
\\Despite the clear and functional benefits of investing in the stock market, 
many Filipinos remain hesitant to do so for the following reasons:
\begin{itemize}
  \item[(a)] The complexities associated with learning the fundamentals 
  of effective stock investing.
  \item[(b)] The time-consuming nature of technical and fundamental 
  analysis, particularly for students and working individuals with limited time.
  \item[(c)] The higher financial risk due to the volatility of the stock market, 
  as well as the potential for emotional decision-making to compromise investments
\end{itemize}
These factors contribute to a lack of confidence and understanding 
among potential investors, making it difficult for them to take advantage of the 
opportunities offered by the stock market.
\\As such the development of the proposed system, shall help to address the following:
\begin{itemize}
  \item[(a)] The lack of free and publicly available 
  stock market prediction systems or tools.
  \item[(b)] The time and resources required to study complex traditional 
  market analysis tools, such as fundamental and technical analysis.
  \item[(c)] The potential for inaccurate market decisions
  leading to significant investment losses.
  \item[(d)] The hesitancy of the Filipino public to begin 
  investing in the Philippine stock market.
\end{itemize}

% Significance of the Study
\section{Significance of the Study}
\label{sec:significance}
The significance of this special problem lies in its potential to develop a 
system that will greatly benefit the stock market, individual investors, and 
the overall economy. The system's contributions to data-driven investing, 
financial protection and management, and economic development will provide 
a valuable resource for investors and help to promote financial stability and growth. 
Additionally, the development of publicly accessible data-driven investing tools 
will enable more Filipinos to participate in the market and take control of their 
own financial future. Overall, this study has the potential to make a meaningful 
impact on the stock market and the economy in the Philippines.
\\Specifically, this study is significant for the following reasons:
\begin{itemize}
  \item[(a)] The development of the alamAPI will provide the 
  following benefits to the Filipino people:
  \subitem 1.	Access to simplified yet accurate information 
  – The proposed system will provide Filipino investors with fast, 
  accurate, and relevant information necessary for effective decision making 
  in the stock market. Using advanced machine learning classifiers, the system 
  will provide users with the two most important pieces of information: 
  which stocks to buy, and which stocks to sell. This simplified investing 
  model will help investors to make informed decisions and navigate the stock 
  market with confidence.
  \subitem 2.	Provide an application interface to facilitate data-driven 
  and wise market decisions – The proposed system will provide users with 
  an intuitive and user-friendly application interface to facilitate data-driven 
  investment decisions, particularly during times when the market is unpredictable 
  or experiencing a downturn. Whereas traditional market analysis tools may not 
  be sufficient to navigate these challenging conditions, the system's advanced 
  machine learning algorithms will provide investors with the insights and guidance 
  they need to make informed and wise decisions. This will help to promote 
  confidence and stability in the market, even during times of uncertainty.
  \subitem 3.	A platform for accessible stock market investment – 
  The proposed system will provide all investors, regardless of their 
  investment knowledge, educational attainment, and societal status, 
  with a platform for participating in the stock market. By offering a 
  simplified yet accurate model for investment decision making, the 
  system will empower users to make informed decisions and invest with confidence. 
  This will help to democratize access to the stock market and promote financial 
  inclusion for all Filipinos.
  \item[(b)] The development of the alamAPI, specifically the Stock Market Price 
  Trend Forecasting System (SMPTF Sys), will provide the following benefits 
  to the future developers or researchers:
  \subitem 1.	Extension of functionality to other financial markets 
  – The proposed system can be easily adapted or expanded to address 
  related problems in other financial markets, such as investing in
  government bonds or personal finance management. 
  This flexibility and versatility will make the system a valuable tool 
  for a wide range of investment and financial management scenarios.
  \subitem 2.	Testing of new trading algorithms and machine learning models – 
  The system provides a platform for introducing and testing new data-driven 
  trading algorithms and machine learning models. This will allow researchers 
  and developers to continually improve the system and keep it at the forefront 
  of data-driven investing technology.
  \subitem 3.	Development of a graphical user interface – 
  To further improve the public accessibility of the system, 
  a user-friendly graphical user interface can be developed as a 
  web or mobile application. This will make the system easy to use 
  and intuitive for all users, regardless of their technical expertise.
  \item[(c)] The development of the alamAPI will help to stimulate economic 
  recovery and development in the country by increasing the number of local 
  investors. As discussed in previous sections, the benefits of the system 
  will encourage more people to invest in the stock market, leading 
  to a multiplier effect that will benefit the economy in several ways. 
  For example, the increased participation in the market will lead to the 
  creation of jobs and a lowering of unemployment rates. Additionally, 
  the influx of capital into the market will drive fast developments and 
  innovations in various industries, and the increased consumer spending 
  that results from successful investing will stimulate economic growth. 
  Overall, the development of the alamAPI will have a positive and 
  far-reaching impact on the economy of the Philippines.
\end{itemize}

% Objectives
\section{Objectives}
\label{sec:objectives}
The main objective of this special problem is to develop a 
system that will make investing easier, more publicly available, 
data-driven, and more approachable to the public by minimizing both the 
time required for stock price trend analysis, and potential financial risk 
by using a predictive model. Specifically, it aims to do following:
\begin{itemize}
  \item[(a)] Develop a RESTful API, which will be referred to as alamAPI, 
  using the combination of Python libraries and MongoDB for the backend 
  services and database, respectively.
  \\Whereas this backend service will be using Python’s 
  FastAPI that will enable different user to connect to 
  the database and collect the information provided by the 
  Stock Market Price Trend Forecasting Model.
  \\Specifically, this will be done by doing the following:
  \subitem 1.	Develop a Data Collector Module (DCM), 
  which will get the historical data every day for 
  the past 200 days every after market close from Mondays to Fridays. 
  The data collected will be eventually processed by the Pre-Database Processor 
  (PDB) with the help of the Preprocessor Utilities Module (PUD). 
  Wherein, DCM, PDB, and PUD will be part of the Preprocessor Module 
  (PPD) of alamAPI.
  \subitem 2.	Develop a database that will store the results provided by the PPD,
  and other essential data about the stock market that is needed to be provided
  in the backend service.
  \subitem 3.	Develop the necessary API endpoints 
  that will provide recommendation on which stocks to buy or sell. 
  Additionally, it provides the general information about these stocks.
  \item[(b)] Develop a Stock Market Price Trend Forecasting Machine Learning Model. 
  Using Dynamic Mode Decomposition (DMD) and Arnaud Legoux Moving Average (ALMA)
   for data preprocessing. 
   Afterwards the preprocessed data will be fed to a Long Short-Term Memory (LSTM)
    Neural Network which will be responsible for predicting future stock prices.
  \item[(c)] Finally, develop a mobile-based test application to showcase 
  the main functionalities of the developed RESTful API.
\end{itemize}

% Scope and Limitations
\section{Scope and Limitations}
\label{sec:scope}
This study is limited only within the companies listed in the 
Philippine Stock Exchange, including the PSE Index itself. 
Wherein, only 20 selected high volume trade stocks from the year 2021 to 2022 
which are has following stock symbols: (1) MEG, (2) JGS, (3) BDO, (4) FGEN, 
(5) ICT, (6) ALI, (7) SMC, (8) TEL, (9) GLO, (10) BLOOM, (11) RLC, (12) MER, 
(13) AC, (14) PGOLD, (15) LTG, (16) MPI, (17) AP, (18) RRHI, (19) URC, and 
(20) PSE Index will be included in the system, instead of the total 286 listed
under the Philippine Stock Exchange, this is because the data directly 
from the Philippine Stock Exchange Inc., is not free, and the free data 
provided by a third-party only allows for 20 requests per day. Also, using 
data scraping tools may prove to be illegal, as it is considered as data theft
because data provided in those websites are for public viewing purposes only 
and are also paid by the companies hosting them.