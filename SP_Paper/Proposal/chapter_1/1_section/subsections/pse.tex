% The Philippine Stock Exchange (PSE)
\subsection{The Philippine Stock Exchange (PSE)}
The Philippine Stock Exchange (PSE), Inc. is the official stock exchange 
market in the Philippines. It is a non-stock company that was incorporated 
in 1992 and manages and operates the stock market in the country. Registered 
individuals can participate in market exchanges on the PSE.
\cite{PSECompanyInfo}.
\vspace{0.5cm}
\\Moreover, the main index of the Philippine Stock Exchange (PSE) 
is the Philippine Stock Exchange Index (PSEI). The PSEI is a market 
capitalization-weighted price index that is based on the 30 largest and 
most actively traded companies on the PSE. These companies are pre-determined 
based on strict criteria, such as liquidity and market capitalization. 
The PSEI is often used as a benchmark for the performance of the overall stock 
market in the Philippines.  
\cite{BSP}
The companies that make up the PSEI are often referred to as blue-chip companies, 
as they are typically large, well-established companies with a history of strong 
financial performance. As of October 2022, there are 286 companies listed on the PSE,
 providing a diverse range of investment opportunities for investors. 
\cite{Fayed2022, PSECompanyList}.