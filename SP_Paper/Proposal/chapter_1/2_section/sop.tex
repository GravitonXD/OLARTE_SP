% Statement of the Problem
\section{Statement of the Problem}
\label{sec:problem}
The Philippines' economic growth is expected to decline in the coming 
years due to the global pandemic, high inflation, and low employment rates. 
\cite{Alegado2022,Canto2022,Reuters2022}.
\\Currently, the lack of free and publicly available stock market predictive systems or
 tools creates a gap in the information available to the public compared to large private 
 individuals or institutions. These large institutions have the resources to spend a significant 
 amount of money on stock market research, giving them a significant advantage in the investing 
 market. This lack of access to the same information puts the public at a disadvantage
 \cite{Kim2022}. 
\\ Furthermore, the lack of publicly available stock market prediction tools can 
lead to unwise investment decisions by individuals, particularly first-time investors, 
resulting in significant losses and discouragement from investing in the stock market. 
This is a significant problem, as the number of local investors in the Philippine Stock Market 
is already quite low, comprising only around 1\% of 
the total population and there also has been a massive decline in foreign investment
 in the Philippines in recent years 
 \cite{BusinessWorld2022}, 
 leading to a corresponding decline in investment volume. 
 As suggested by the study of \citeA{Balaba2017}, 
 this is expected to have a negative multiplier effect on the country's 
 economic development in the future. 
\\ Hence, the development of a publicly available, easy-to-use, and
 accurate stock market price trend prediction system could help to 
 reduce the information gap and level the playing field for individual investors. 
 By providing the public with fast and reliable information, this system could help to
  increase transparency and fairness in the stock market, leading to more informed 
  and confident investing decisions and ultimately a more stable and prosperous market.
   Additionally, such a system could help to increase the participation of individual 
   investors in the market, leading to a more diverse and stable market overall.
\cite{Statista2022,POPCOMM2021}.
\\Despite the clear and functional benefits of investing in the stock market, 
many Filipinos remain hesitant to do so for the following reasons:
\begin{itemize}
  \item[(a)] The complexities associated with learning the fundamentals 
  of effective stock investing.
  \item[(b)] The time-consuming nature of technical and fundamental 
  analysis, particularly for students and working individuals with limited time.
  \item[(c)] The higher financial risk due to the volatility of the stock market, 
  as well as the potential for emotional decision-making to compromise investments
\end{itemize}
These factors contribute to a lack of confidence and understanding 
among potential investors, making it difficult for them to take advantage of the 
opportunities offered by the stock market.
\\As such the development of the proposed system, shall help to address the following:
\begin{itemize}
  \item[(a)] The lack of free and publicly available 
  stock market prediction systems or tools.
  \item[(b)] The time and resources required to study complex traditional 
  market analysis tools, such as fundamental and technical analysis.
  \item[(c)] The potential for inaccurate market decisions
  leading to significant investment losses.
  \item[(d)] The hesitancy of the Filipino public to begin 
  investing in the Philippine stock market.
\end{itemize}