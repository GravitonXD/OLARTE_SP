% Objectives
\section{Objectives}
\label{sec:objectives}
The main objective of this special problem is to develop a 
system that will make investing easier, more publicly available, 
data-driven, and more approachable to the public by minimizing both the 
time required for stock price trend analysis, and potential financial risk 
by using a predictive model. Specifically, it aims to do following:
\begin{itemize}
  \item[(a)] Develop a RESTful API, which will be referred to as alamAPI, 
  using the combination of Python libraries and MongoDB for the backend 
  services and database, respectively.
  \vspace{0.5cm}
  \\Whereas this backend service will be using Python’s 
  FastAPI that will enable different user to connect to 
  the database and collect the information provided by the 
  Stock Market Price Trend Forecasting Model.
  \vspace{0.5cm}
  \\Specifically, this will be done by doing the following:
  \subitem 1.	Develop a Data Collector Module (DCM), 
  which will get the historical data every day for 
  the past 200 days every after market close from Mondays to Fridays. 
  The data collected will be eventually processed by the Pre-Database Processor 
  (PDP) with the help of the Preprocessor Utilities Module (PUM). 
  Wherein, DCM, PDP, and PUM will be part of the Preprocessor Module 
  (PPM) of alamAPI.
  \subitem 2.	Develop a database that will store the results provided by the PPD,
  and other essential data about the stock market that is needed to be provided
  in the backend service.
  \subitem 3.	Develop the necessary API endpoints 
  that will provide recommendation on which stocks to buy or sell. 
  Additionally, it provides the general information about these stocks.
  \item[(b)] Develop the following Stock Market Price Trend Forecasting 
  Machine Learning Models:
    \subitem 1.	Model A (DMD-LSTM) – Utilize the dynamic modes in DMD as a 
    parameter to the LSTM model.
    \subitem 2. Model B (ARIALMA) – Modify ARIMA by using 
    the optimized parameters for ALMA instead of the traditional MAs.
  \item[(c)] Finally, develop a mobile-based test application to showcase 
  the main functionalities of the developed RESTful API.
\end{itemize}