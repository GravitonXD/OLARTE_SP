\subsection{Software Requirements}
\label{sec:software_requirements}
\begin{itemize}
    % Python
    \item[(a)] Python (version 3.11.x) – this will serve as the main programming 
    language for the development of the different components of alamAPI, more 
    specifically the following libraries will be used:
        \begin{itemize}
            \item[\ding{108}] For the development of the API and Database ODM
                \begin{itemize}
                    \item[\ding{109}] FastAPI (version 0.85.0) – This is a library primarily 
                    used for building modern, fast, and high-performing web framework APIs 
                    \cite{Tiangolo}.
                    This will be utilized in the development of the project because of its 
                    (1) ease of utilization; 
                    (2) fast implementation; 
                    (3) high-performance; 
                    (4) built-in robust API documentation; and 
                    (5) high scalability.
                    \item[\ding{109}] mongoengine (version 0.24.2) – This is a library developed 
                    as an Object-Document Mapper, which lets Python connect and work with MongoDB 
                    \cite{MongoEngine}
                    This will be used in the alamAPI to connect the API endpoints to the 
                    MongoDB database.
                    \item[\ding{109}] json (pre-installed) – This is a python library that 
                    can transform Python dictionary into json object, and vice versa. 
                    This will be used in the development of alamAPI for parsing and conversion of 
                    the data from the API and to the MongoDB database through an ODM.
                    \item[\ding{109}] datetime (pre-installed) – This python library is used for creating 
                    a datatime object, which as the name suggests is an object that contains the date 
                    and time information. This will be used in the development to keep track with all 
                    the processes that is happening in the system through a date and time logs.
                    \item[\ding{109}] os (pre-installed) – This is a python library that enables 
                    the user to do operations in the operating system such as creating directories, 
                    files, accessing operating system information, etc. This will be used to access 
                    the operating system’s environment variables, and to help in other OS-based functions.
                \end{itemize}
            \item[\ding{108}] For the pre-processor (data collector)
                \begin{itemize}
                    \item[\ding{109}] requests (version 2.28.1) – This library allows the user to create web 
                    requests to an external or internal servers. This will be used to connect and collect 
                    the current EOD market data from the third-party market historical data provider: EODHD.
                    \\ EODHD – A third-party market fundamental and historical data APIs provider
                    \cite{EODHD}.
                \end{itemize}
            \item[\ding{108}] For the pre-processor (machine learning processor):
            \\ \textit{Note that these libraries will also be used in the development of the machine learning model.}
                \begin{itemize}
                    \item[\ding{109}] pickle (pre-installed) – This allows an object to be 
                    saved and reloaded as a variable in Python, as such this will be used to 
                    save the machine learning developed and be utilized to process the new and 
                    updated data provided by the data collector.
                    \item[\ding{109}] joblib (pre-installed) – xxx
                    \item[\ding{109}] numpy - xxx
                    \item[\ding{109}] pandas - xxx
                    \item[\ding{109}] sklearn - xxx
                    \item[\ding{109}] tensorflow - xxx
                    \item[\ding{109}] matplotlib - xxx
                    \item[\ding{109}] seaborn - xxx
                \end{itemize}
            \item[(b)] MongoDB – This will be used as the non-relational (document-based) database, 
            that will hold the stock information, stocks to buy, and stocks to sell.
            \item[(c)] Jupyter Notebook – This will be used during the training and testing of 
            the machine learning model that will be developed as part of the alamAPI.
            \item[(d)] CRON – A Linux-based scheduler. This will be used in the system to set a 
            schedule for the historical data collection and processing for each market end-of-day 
            (EOD) every 5 PM from Mondays to Fridays. Moreover, the scheduler is part of the 
            pre-processor module of the system.
            \item[(e)] Docker – This is a very useful tool to creating containers, 
            whereas a container contains a code and all its dependencies in one standard unit of software, 
            which can be run in different machines regardless of its difference from the development machine 
            used \cite{Docker}.
            As such this will be used to create containers for each of the component of alamAPI, 
            to enable it to run in different deployment machines.
            \item[(f)] Docker-compose – In order to run multiple containers at once, 
            docker-compose will be used. This will be further discussed in the Container Diagram 
            section of this chapter.
            \item[(g)] Dart and Flutter - xxx
            \item[(h)] Git - xxx
        \end{itemize}
\end{itemize}