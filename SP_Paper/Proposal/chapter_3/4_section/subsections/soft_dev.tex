\subsection{Software Development Process}
\label{subsec:soft_dev}
Due to the expected heavy time constraints of the development of the system, 
the author of this paper decided to follow an Agile Software Development Process, 
primarily it will be using Agile Sprints for an efficient time management 
during the whole software development process. Wherein the following are the 
list of Sprints and sub-activities that will be followed are shown in the 
Table below:

% TABLE: SPRINTS AND SUB-ACTIVITIES
\begin{longtable}{|c|l|l|}
    \caption{Summary of Sprints and Target Activities}
    \label{summary-sprints}\\
    \hline
    % Header
    \textbf{Sprint Number} & 
    \multicolumn{1}{c|}{\textbf{Target Activities}} & 
    \multicolumn{1}{c|}{\textbf{Allotted Time$^1$}} \\ \hline
    \endfirsthead
    % Continued Header
    \multicolumn{3}{c}%
    {{\bfseries Table \thetable\ continued from previous page}} \\
    \hline
    \textbf{Sprint Number} & 
    \multicolumn{1}{c|}{\textbf{Target Activities}} & 
    \multicolumn{1}{c|}{\textbf{Allotted Time$^1$}}\\ \hline
    \endhead
    % Row 1
    1 &
    \begin{tabular}{p{0.35\textwidth}}
        \textbf{Main Activity:} System Planning and Evaluation \\
        \vspace{0.5cm}
        \textbf{Sub-Activities:}
        \begin{itemize}
            \item Topic Proposal 
            \item Drafting of Chapters 1 to 3 for the Special Problem Proposal
            \item System Architecture and User Requirement Analysis
        \end{itemize}
    \end{tabular} &
    \begin{tabular}{p{0.20\textwidth}}
        \textbf{12 Weeks}
        \\Start: \\September 15, 2022
        \\End: \\December 9, 2022
    \end{tabular} \\ \hline
    % Row 2
    2 &
    \begin{tabular}{p{0.35\textwidth}}
        \textbf{Main Activity:} System Prototyping \\
        \vspace{0.5cm}
        \textbf{Sub-Activities:}
        \begin{itemize}
            \item Build the different component of the alamAPI as 
            indicated in the top-level overview diagram of the system, 
            the following prototype will be developed:
            \subitem[1.] API endpoints
            \subitem[2.] Database
            \subitem[3.] Preprocessor
            \item Testing of the build prototype. This also 
            include creating unit test cases for each component.
            \item Initial Documentations, this will be done 
            inside the GitHub repository.
        \end{itemize}
    \end{tabular} &
    \begin{tabular}{p{0.20\textwidth}}
        \textbf{12 Weeks}
        \\Start: \\September 30, 2022
        \\End: \\April 3, 2023
    \end{tabular} \\ \hline
    % Row 3
    3 &
    \begin{tabular}{p{0.35\textwidth}}
        \textbf{Main Activity:} Machine Learning Model Training, 
        Testing, and Evaluation \\
        \vspace{0.5cm}
        \textbf{Sub-Activities:}
        \begin{itemize}
            \item Collection of Historical Data, outside the 
            Data Collector module of the system. As the full 
            data will be needed for each stock for the training, 
            rather the 200-day only historical data. Whereas the last 
            date on the market data should be January 13, 2023.
            \item Development of the Machine Learning Model. 
            This includes data standardization, data splitting, 
            and data training.
            \item Machine Learning model testing and evaluation.
            \item Revision of Chapter 1-3, in preparation 
            for the final paper submission.
        \end{itemize}
    \end{tabular} &
    \begin{tabular}{p{0.20\textwidth}}
        \textbf{10 Weeks}
        \\Start: \\January 15, 2023
        \\End: \\March 30, 2023
    \end{tabular} \\ \hline
    % Row 4
    4 &
    \begin{tabular}{p{0.35\textwidth}}
        \textbf{Main Activity:} Integration of Machine Learning Model 
        to the alamAPI and Additional Data Collection \\
        \vspace{0.5cm}
        \textbf{Sub-Activities:}
        \begin{itemize}
            \item Testing and Evaluation of alamAPI with the 
            integration of the Machine Learning Model.
            \item System Testing, this will be done to verify 
            the functionality of the whole system, given a test 
            deployment environment. Moreover, this will be done 
            in a span of 4 weeks
            \item Drafting of Chapter 4 and 5
        \end{itemize}
    \end{tabular} &
    \begin{tabular}{p{0.20\textwidth}}
        \textbf{6 Weeks}
        \\Start: \\March 31, 2023
        \\End: \\May 12, 2023
    \end{tabular} \\ \hline
    % Row 5
    5$^2$ &
    \begin{tabular}{p{0.35\textwidth}}
        \textbf{Main Activity:} System Documentation \\
        \vspace{0.5cm}
        \textbf{Sub-Activities:}
        \begin{itemize}
            \item Updating and Finalization of Documentations 
            included in the GitHub Repository.
            \item Writing of the results, discussions, conclusions, 
            and recommendations for Chapter 4 – 5
            \item Special problem paper revisions
            \item Start the development of the test application 
            (for showcasing of the system features)
        \end{itemize}
    \end{tabular} &
    \begin{tabular}{p{0.20\textwidth}}
        \textbf{6 Weeks}
        \\Start: \\April 14, 2023
        \\End: \\May 26, 2023
    \end{tabular} \\ \hline
    % Row 6
    6$^2$ &
    \begin{tabular}{p{0.35\textwidth}}
        \textbf{Main Activity:} Preparation for Final 
        Defense and System Presentation \\
        \vspace{0.5cm}
        \textbf{Sub-Activities:}
        \begin{itemize}
            \item Finalization of the mobile-based test application
            \item Revisions and Finalization of the special problem paper.
            \item Creation of presentation slide deck for the presentation 
            of the special problem.
        \end{itemize}
    \end{tabular} &
    \begin{tabular}{p{0.20\textwidth}}
        \textbf{3 Weeks}
        \\Start: \\May 27, 2023
        \\End: \\June 17, 2023
    \end{tabular} \\ \hline
    \caption*{
        1. Start and End Dates are based on 
        the University’s Academic Calendar 
        and the Schedule provided by the 
        Special Problem Adviser.\\
        2. Sprints 5 and 6 are no longer part of the 
        actual system development but is still included as 
        a basis for the Gantt chart. Moreover, these 
        activities can still be considered as part of the 
        documentation process.
    }
\end{longtable}

\vspace{0.5cm}
From the given table above, it is 
shown that there is a total of 39 weeks; from September 15, 2022, 
to June 17, 2023, however it must be noted that an additional 1 week 
was added to each sprint’s allotted time to compensate for any unforeseen 
events during each sprint.
\vspace{0.5cm}
\\ It should also be noted that Sprint 1 and Sprint 2 overlaps as 
the development of the prototype will start at Week 3, this will 
be possible as there will already be an initial system design to be followed, 
and any changes made during Sprint 1 can easily be adjusted to the creation of 
the prototype of the system in Sprint 2. This is also the case for Sprints 4 and 5, 
since their activities overlaps with each other, such that there are things in 
Sprint 4 that are unsupervised, hence, to better manage the time it is reasonable 
to start the activities of Sprint 5 along side the later parts of Sprint 4.