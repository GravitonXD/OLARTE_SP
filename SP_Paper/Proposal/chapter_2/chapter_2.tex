\chapter{Review of Related Works and Literature}
\label{chap:lit_review}
One of the challenges facing investors in the Philippine Stock Market 
is the limited availability of resources and tools for 
making market decisions. In contrast, other countries have begun 
implementing machine learning techniques for stock market prediction and analysis, 
which allows for more accurate decision-making and reduces the risk of poor 
investment outcomes. As a result, these countries are likely to experience 
better returns on their investments.
\\In this literature review, the following general topics are reviewed, 
discussed, and synthesized: (a) Integration of Machine Learning based 
Trading Algorithms; and (b) Utilization of Dynamic Mode Decomposition on the 
Financial markets.

% Integration of Machine Learning based Trading Algorithms
\section{Integration of Machine Learning based \\Trading Algorithms}
\label{sec:integration_ml}
Stock market analysis is crucial for effective risk management. 
This involves using various methods, such as technical and fundamental 
analysis, to make informed decisions for investors and traders. 
In recent years, the growth of computing power and resources has 
led to the increasing use of machine learning techniques for stock market 
prediction and analysis. These advances help companies better predict upcoming 
market trends and make more informed decisions.
\\The integration of machine learning algorithms in the stock market is growing, 
as investors and traders increasingly rely on fast and accurate market information 
to reduce potential risks and make better decisions. These algorithms allow for 
more efficient analysis of market data, leading to more informed decisions and 
improved investment outcomes
\cite{Obthong2020}.

\subsection{Comparison of Machine Learning Models in Stock Market Predictions}
\label{subsec:comparison_ml}
To have a better grasp in the accuracy of the different models 
used in algorithmic trading it is essential that different models are 
compared against each other.
\subsubsection{Combination of Computational Efficient Functional Link Artificial Neural Network (CEFLANN) and Traditional Technical Analysis}
\label{subsubsec:ceflann}
This hybrid model combines a classification-based model: CEFLANN and 
the traditional technical analysis to create a stock trading framework 
\citeA{Dash2016}, which the results show a profit of 24.29\%.
\subsubsection{Deep Long Short-Term Neural Network (LSTM) with Embedded Layer}
\label{subsubsec:lstm}
In one of the models developed by \citeA{Pang2020}, 
it shows that by adding an embedded layer to the LSTM it yields to a 
stock market price prediction accuracy of 57.2\%. 
However, its accuracy dips to 52.4\% when the model is applied to individual stocks.
\subsubsection{LSTM with Automatic Encoder}
\label{subsubsec:lstm_autoencoder}
As part of the second model developed by \citeA{Pang2020}, this model 
shows a slightly inaccurate stock market prediction, by only having 
a measured accuracy of 56.9\%. 
However, compared to the first model developed by the group this is 0.1\%
more effective for individual stocks.
\subsubsection{Optimal Deep Learning (ODL)}
\label{subsubsec:odl}
In the study conducted by \citeA{Agrawal2019} they have created a 
stock price prediction model using an Optimal Deep Learning (ODL) 
which combine the concepts of Correlation-Tensor and an Optimal LSTM algorithm. 
Whereas their results show a mean and highest accuracy of the model as 59.24\% 
and 65.64\%.
\subsubsection{NMC-BERT-LSTM-DQN-X Algorithm}
\label{subsubsec:nmc_bert_lstm_dqn_x}
More recently, a team have applied a combination of three models for 
forecasting the market trends. Namely, (1) Non-stationary Markov Chain (NMC), 
(2) Bidirectional Encoder Representations from Transformers (BERT), 
(3) Long Short-Term Memory (LSTM). Wherein their model shows an accuracy of 61.77\%. 
Furthermore, the team also mentioned that the model produces 29.25\% 
annual return on investment, with a maximum losses rating of -8.29\% 
\cite{Liu2022}.

% Utilization of Dynamic Mode Decomposition on the Financial markets
\section{Utilization of Dynamic Mode Decomposition\\(DMD) on the Financial Markets}
\label{sec:utilization_dmd}
Dynamic Mode Decomposition (DMD) as an emerging data-driven technique 
which allows spatial-temporal pattern recognition from a complex set of 
data and was first introduced in the field of fluid mechanics by 
\cite{SCHMID2010}.
\subsection{Chronological Utilization of DMD in the Financial Markets}
\label{subsec:chronological_utilization_dmd}
In \citeyear{Mann2015} Mann and Kutz proved that DMD can be used as data-driven analytics 
on the financial market data. Wherein, DMD allows a predictive assessment of 
the market dynamics, which helps in the capitalization of stock market 
strategies and decisions to be applied.
\subsubsection{Utilization of DMD for Determining the Cyclic Behavior in the Stock Market (2016)}
\label{subsubsec:dmd_cyclic_behavior}
By utilizing the reproducible Koopman modes it made it possible to have extracted 
four cyclic variations (also reproducible modes) in the stock market, which 
were previously unknown and have persisted since the 1870s’ 
global economic crisis
\cite{Hua2016,Williamson2015}.
\subsubsection{Utilization of DMD as part of an Algorithmic Trading Strategies for the Turkish Stock Market (2015 and 2017)}
\label{subsubsec:dmd_algorithmic_trading}
The study of \citeA{Mann2015} in the utilization of DMD for financial stock market 
prediction has become the foundation of the study by \citeA{Savas2017} 
on the algorithmic trading strategies with Dynamic Mode Decomposition 
for the Turkish Stock Market. Wherein, based on their results they found out 
that the timing of DMD analysis was not significantly accurate, as such they 
have used a simple moving average with genetic algorithm to improve the market 
timing of DMD, which prevents 80\% of the false trade signals.
\\Furthermore, this also shows that DMD is an effective alpha 
model that is easy to implement and use for any algorithmic trading 
strategy, and the addition of technical analysis tools can further improve 
its capabilities, especially on the predictive temporal side of the data.
\subsubsection{Utilization of DMD-based Trading Strategy in the Chinese Stock Market (2016)}
\label{subsubsec:dmd_chinese_stock_market}
In the study by \citeA{Cui2016}, they have found that DMD was able to capture the 
dynamic patterns of the Chinese Stock Market, especially in a sideway trending market.
\\Their study also shows that the predictive ability of DMD can effectively 
model the behavior of the Chinese Stock Market, even if there are no 
clear trends that can be observed.
\subsubsection{Utilization of Adaptive Elastic DMD to Improve Momentum Strategies (2021)}
\label{subsubsec:dmd_adaptive_elastic}
A study by \citeA{Uchiyama2021}, using Adaptive Elastic Dynamic Mode Decomposition 
(AEDMD) shows that they were able to estimate the market trend, and were able to 
demonstrate that the approach is better than existing momentum strategy which 
are only based on simple past trends.

% Synthesis of Literature Review
\section{Synthesis}
\label{sec:synthesis}
Fast and accurate market information is an essential tool 
for stock market participants. In recent years, the development 
of machine learning models for the financial markets, such as stocks, 
has proven to be increasingly effective in predicting future stock prices and trends. 
The use of Dynamic Mode Decomposition (DMD) in the stock market has also been shown 
to be effective in predicting stock price trends. The simplicity and elegance of 
the Koopman Decomposition Operator make it an ideal basis for the development of 
a Stock Market Price Trend Forecasting System (SMPTF System).
\\These studies are crucial for the development of the alamAPI to 
provide investors with fast and accurate information about which stocks are 
likely to go up or down, allowing them to make more informed decisions about 
buying or selling those stocks.
\\In addition to the potential benefits for investors and traders, the 
implementation of machine learning techniques in the stock market can also 
help improve market efficiency and reduce the risk of market manipulation. 
By providing a more accurate and comprehensive view of market trends, 
these techniques can help ensure that prices reflect the true value of 
stocks and other assets, leading to more stable and fair market conditions.